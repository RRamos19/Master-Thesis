% 
%  chapter1.tex
%  ThesisISEL
%  
%  Created by Matilde Pós-de-Mina Pato on 2012/10/09.
%
\chapter{Introdução}
\label{capitulo:introdução}

Os problemas de agendamento são uma classe de problemas de otimização combinatória que surgem em diversas áreas, sendo fundamentais para melhorar a eficiência e a alocação de recursos em cenários reais. A essência do problema de agendamento está na organização de um conjunto de tarefas (ou eventos) num intervalo de tempo, de forma a atender a um conjunto de restrições e, em muitos casos, otimizar um ou mais critérios. Esses problemas aparecem em contextos tão variados quanto educação, logística, saúde, produção industrial, transportes, entre outros.

A elaboração anual ou semestral de horários escolares é tipicamente feita de forma manual. Devido à existência de múltiplas soluções possíveis, esta atividade é muito demorada e complexa. Em certos casos, pode não ser possível criar um horário devido ao número de restrições presentes.

As restrições podem ser rígidas ou flexíveis. Restrições rígidas são de cumprimento obrigatório não sendo possível criar um horário caso a restrição seja violada. Restrições flexíveis podem não ser cumpridas apesar de haver um incentivo para o cumprimento das mesmas. A qualidade de uma solução de horário depende da quantidade de restrições flexíveis violadas.

Os problemas de agendamento podem ser abordados por meio de duas subcategorias:

\begin{compactitem}
    \item Agendamento de Disciplinas Baseado no Currículo ou, em inglês, \gls{cb-ctt}.
    \item Agendamento de Disciplinas Após Inscrição ou, em inglês, \gls{pe-ctt}.
\end{compactitem}

A formulação \gls{cb-ctt} centra-se no planeamento de horários com base no currículo escolar, organizando disciplinas que compõem programas de estudo estruturados. Este tipo de abordagem é amplamente utilizado no ensino primário e secundário, mas também pode ser aplicado no ensino superior.

Por outro lado, a formulação \gls{pe-ctt} foca-se no planeamento de horários considerando as inscrições dos estudantes em disciplinas específicas. Esta abordagem é particularmente relevante no ensino superior, onde os estudantes têm maior liberdade na escolha das disciplinas que desejam frequentar.

\section{Objetivos}

O projeto tem como objetivo principal o desenvolvimento de uma aplicação com interface gráfica que permita a criação de horários, a partir de uma configuração inicial acerca do ciclo (licenciatura, mestrado) do horário e os dados relativos às salas, disciplinas e professores. Para a criação dos horários será utilizada a meta-heurística Têmpera simulada ou, em inglês, \gls{sa} e será seguido o padrão da \gls{itc}~2019 \cite{itc2019-Website}.

A aplicação será desenvolvida de modo a ser utilizada pelo \gls{isel}, seguindo as normas estabelecidas pela instituição. Para a formulação do problema de agendamento, será adotada a abordagem \gls{cb-ctt}, uma vez que a inscrição dos estudantes ocorre após a criação dos horários.

Pretende-se a utilização do padrão da \gls{itc}~2019 de modo a simplificar o armazenamento e representação dos dados num formato definido e amplamente utilizado. A utilização de um padrão definido possui algumas vantagens, tais como: redução do tempo de implementação, fiabilidade e qualidade, facilidade de manutenção e minimização de erros de implementação.

O projeto terá um foco na otimização do processo de geração de horários e haverá uma fase de descoberta dos melhores parâmetros do algoritmo \gls{sa} para este caso em específico, comparando os tempos de execução e os resultados obtidos.

Os dados utilizados por parte do projeto devem ser guardados de forma persistente para que os seus utilizadores não sejam obrigados a fornecer os mesmos dados entre utilizações. Será criada uma base de dados para armazenar os dados fornecidos de forma organizada e normalizada.

\begin{figure}[htbp]
    \centering
    \includegraphics[width=\linewidth]{diagrama-geral}
    \caption{Diagrama geral da aplicação a desenvolver.}
    \label{fig:diagrama-geral}
\end{figure}

A Figura~\ref{fig:diagrama-geral} apresenta o diagrama geral da aplicação. Na primeira utilização, o utilizador deve introduzir os dados referentes aos cursos, disciplinas e professores, seguindo o padrão estabelecido pela \gls{itc}~2019.

Após a inserção e processamento dos dados, estes são exibidos ao utilizador através da interface gráfica. Os dados introduzidos são armazenados numa base de dados, garantindo a sua persistência entre diferentes utilizações da aplicação.

Quando o utilizador inicia o processo de geração de horários, a aplicação executa o algoritmo responsável pela criação dos horários. Durante este processo, uma barra de progresso é exibida para fornecer informação sobre o estado de execução da operação.

%TODO: Adicionar informação acerca dos parâmetros de entrada

Por fim, o utilizador pode optar por exportar os dados armazenados na base de dados. A exportação pode ser realizada nos formatos \gls{xml} e \gls{csv} para todos os dados, sendo os formatos \gls{png} e \gls{pdf} exclusivos para os horários.

A exportação dos dados no formato \gls{xml} seguirá o padrão estabelecido pela \gls{itc}~2019. No formato \gls{csv}, serão exportadas as tabelas presentes na base de dados. Já nos formatos \gls{png} e \gls{pdf}, serão exportados exclusivamente os horários, sendo apresentados por meio de uma imagem típica de horário. No formato \gls{pdf}, a imagem será gerada como um gráfico vetorial.

\section{Organização do documento}

O restante relatório está organizado em dois capítulos.

O Capítulo~\ref{capitulo:estado-arte} tem como objetivo detalhar a compreensão do problema ao explorar abordagens e técnicas descritas na literatura. A análise começa com estudos relevantes a partir do ano de 1996, fornecendo uma perspetiva histórica das soluções propostas. Em seguida, são destacados artigos que fazem uso do algoritmo que será utilizado neste projeto, o \gls{sa}, permitindo compreender a sua aplicação em cenários semelhantes e o impacto das suas características na resolução do problema em questão.

No Capítulo~\ref{capitulo:trabalho-realizado-e-por-realizar}, são detalhados os requisitos funcionais e não funcionais do projeto, juntamente com uma descrição das atividades e avanços realizados até ao momento. Este capítulo resume o trabalho desenvolvido e apresenta as estratégias adotadas para alcançar os resultados pretendidos, bem como o respetivo planeamento.