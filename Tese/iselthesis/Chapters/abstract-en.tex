%!TEX root = ../template.tex
%%%%%%%%%%%%%%%%%%%%%%%%%%%%%%%%%%%%%%%%%%%%%%%%%%%%%%%%%%%%%%%%%%%%
%% abstract-en.tex
%% ISEL thesis document file
%%
%% Abstract in English
%%%%%%%%%%%%%%%%%%%%%%%%%%%%%%%%%%%%%%%%%%%%%%%%%%%%%%%%%%%%%%%%%%%%

The thesis focuses on the problem of scheduling university courses. It is a complex problem due to the immense number of possibilities, which grow exponentially with the amount of available data. This problem is common to all higher education institutions and, due to its complexity, its resolution can be very time-consuming, with no guarantee that the resulting solution will be the most efficient or optimized.

The main objective of this project is to develop a graphical application that allows visualization of the available data and the generated solutions. The data to be used must comply with the format defined by the International Timetabling Competition 2019. Through a simple interface, the application will allow, with a single click, the automatic generation of a timetable for the selected course.

The application aims to significantly reduce the complexity associated with creating timetables by fully automating the process. This approach aims to alleviate the workload of faculty at the beginning of each academic year, providing a faster, more effective, and user-friendly solution.

% Palavras-chave do resumo em Inglês
\begin{keywords}
Course timetabling, Higher education, Optimization problem, International Timetabling Competition, Graphical application, Automatic timetable generation
\end{keywords} 
