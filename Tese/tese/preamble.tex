\usepackage[utf8]{inputenc}
\usepackage[T1]{fontenc}
\usepackage[portuguese]{babel}
\usepackage{helvet}
\renewcommand{\familydefault}{\sfdefault}
\usepackage{geometry}
\geometry{a4paper, top=2.5cm, bottom=2.5cm, left=2.5cm, right=2.5cm}
\usepackage{setspace}
\usepackage{graphicx}
\usepackage{titlesec}
\usepackage{tocloft}
\usepackage{csquotes}
\usepackage{xcolor}
\usepackage{xurl}
\usepackage{hyperref}
\usepackage{paralist}
\usepackage{listings}
\usepackage[style=numeric-comp,
            sortcites=true,
            sorting=nyt,
            backref=true,
            maxbibnames=99,
            giveninits=true,
            hyperref=true]{biblatex}
\addbibresource{bibliography.bib}
\usepackage{glossaries}
\setglossarystyle{super}
\usepackage{float}
\usepackage{booktabs}
\usepackage{makecell} % Tabelas com células em várias linhas. Controlo vertical e alinhamento horizontal
\usepackage{amsmath}  % Alinhamento de equações matemáticas
\usepackage{amssymb}  % Simbolos
\usepackage{pgfgantt} % Cria diagramas de gantt de alta qualidade
\def\keywords{%
    \par\vskip\baselineskip\noindent{\bfseries}%
}
\def\endkeywords{~\\[2ex]\rule{\textwidth}{0.2mm}}

% Títulos semelhantes ao template
\titleformat
{\chapter}                                        % command
[display]                                         % shape
{\bfseries\fontsize{17}{20}\selectfont} % format
{}                       % label
{15pt}                                            % sep
{\centering}                                                % before-code
[]                                                % after-code

 \lstset{frame=tblr,
  breaklines=true,
  showstringspaces=false,
  columns=flexible,
  numbers=none,
  tabsize=3,
  captionpos=b
}

\definecolor{dkred}{RGB}{108, 20, 19}
\definecolor{beige}{RGB}{189, 150, 45}
\definecolor{palegreen}{RGB}{82, 125, 2}

\lstdefinelanguage{XML}
{
  morestring=[b]",
  morestring=[s]{>}{<},
  morecomment=[s]{<?}{?>},
  stringstyle=\color{dkred},
  identifierstyle=\color{beige},
  moredelim=[s][\color{palegreen}]{\ }{=}
}

\addto\captionsportuguese{ % Altera Conteúdo para Índice
    \renewcommand{\contentsname}{Índice}
}

\makeglossaries
%\chapter*{Acronyms} 
%================SYMBOLS & ACRONYMS =================%

\newacronym{sa}{SA}{\textit{Simulated Annealing}}

\newacronym{itc}{ITC}{\textit{International Timetabling Competition}}

\newacronym{ga}{GA}{\textit{Genetic Algorithm}}

\newacronym{uctp}{UCTP}{\textit{University Course Timetabling Problem}}

\newacronym{ts}{TS}{\textit{Tabu Search}}

\newacronym{np}{NP}{\textit{Non-deterministic Polynomial time}}

\newacronym{xml}{XML}{\textit{Extensible Markup Language}}

\newacronym{csv}{CSV}{\textit{Comma-Separated Values}}

\newacronym{pdf}{PDF}{\textit{Portable Document Format}}

\newacronym{png}{PNG}{\textit{Portable Network Graphics}}

\newacronym{cb-ctt}{CB-CTT}{\textit{Curriculum-Based Course Timetabling}}

\newacronym{pe-ctt}{PE-CTT}{\textit{Post-Enrollment Course Timetabling}}

\newacronym{isel}{ISEL}{Instituto Superior de Engenharia de Lisboa}

\newacronym{feup}{FEUP}{Faculdade de Engenharia da Universidade do Porto}

\newacronym{jdbc}{JDBC}{\textit{Java Database Connectivity}}

\newacronym{sql}{SQL}{\textit{Structured Query Language}}

\newacronym{ml}{ML}{\textit{MoveLecture}}

\newacronym{sl}{SL}{\textit{SwapLectures}}

\newacronym{sr}{SR}{\textit{Swap Rate}}
\input{Capitulos/glossary}

\renewcommand\lstlistlistingname{Índice de Listagens}
\renewcommand\lstlistingname{Listagem}

\setlength{\parindent}{0pt}

\onehalfspacing