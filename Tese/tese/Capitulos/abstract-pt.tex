{\centering\bfseries\fontsize{17}{24}\selectfont Ferramenta para Geração de Horários de Cursos com Restrições \par}
\vspace{.5cm}

{\centering\bfseries\fontsize{17}{24}\selectfont Resumo \par}
\vspace{1cm}

A tese centra-se no problema de agendamento de cursos universitários. É um problema complexo devido à imensidade de possibilidades, que crescem exponencialmente com a quantidade de dados disponíveis. Este problema é transversal a todas as instituições de ensino superior e, devido à sua complexidade, a sua resolução pode ser muito demorada, sem a garantia de que a solução obtida seja a mais eficiente ou otimizada.

Este trabalho de projeto tem como principal objetivo o desenvolvimento de uma aplicação gráfica que permita visualizar os dados disponíveis e as soluções geradas. Os dados a utilizar deverão estar em conformidade com o formato definido pela Competição Internacional de Elaboração de Horários (\textit{International Timetabling Competition}) 2019. Através de uma interface simples, a aplicação permitirá, com um único clique, gerar automaticamente um horário para o curso selecionado.

A aplicação pretende reduzir significativamente a complexidade associada à criação de horários, automatizando por completo o processo. Esta abordagem visa aliviar a carga de trabalho dos docentes no início de cada ano letivo, proporcionando uma solução mais rápida, eficaz e de fácil utilização.

\vfill

% Palavras-chave do resumo em Português
Palavras-chave: Agendamento de cursos, Ensino superior, Problemas de otimização, Competição Internacional de Elaboração de Horários, Aplicação gráfica, Geração automática de horários
% to add an extra black line