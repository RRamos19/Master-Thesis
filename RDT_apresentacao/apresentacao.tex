\documentclass[11]{beamer}
\usepackage{hyperref}
\usepackage{graphicx} % Allows including images
\usepackage{booktabs} % Allows the use of \toprule, \midrule and \bottomrule in tables
\usepackage{pgfgantt}
\usepackage[portuguese]{babel}
\usepackage{ragged2e} % Justificar texto

% Definições do tema
\usetheme{copenhagen} % Pode ajustar para outro tema se necessário
\usecolortheme{default}
\setbeamertemplate{navigation symbols}{} % Remove os símbolos de navegação
\setbeamertemplate{headline}{} % Remove o cabeçalho

% Informação da capa:
\title{Ferramenta para Geração de Horários de Cursos com Restrições}
\subtitle{Apresentação do Relatório de Desenvolvimento do Trabalho}

\author[Ricardo Ramos]{Ricardo Alexandre Alves Ramos}

\institute[ISEL]{Instituto Superior de Engenharia de Lisboa}

%\date{Março 01, 2025}
\date{\today}

\institute{
    Instituto Superior de Engenharia de Lisboa \\ 
    Departamento de Engenharia Eletrónica e de Telecomunicações e Computadores \\ 
    Mestrado em Engenharia Informática e de Computadores
}

\begin{document}
    %----------------------------------------------------------------------------------------
    %    PRESENTATION SLIDES
    %----------------------------------------------------------------------------------------

    % Slide de título
    \begin{frame}
        \begin{center}
            \begin{beamercolorbox}[rounded=true, shadow=true, sep=10pt, center, wd=\linewidth]{title}
                \color{white} \usebeamerfont{title} \textbf{\inserttitle} \\
                \usebeamerfont{subtitle} \insertsubtitle
            \end{beamercolorbox}
            
            \vspace{1em}
            {\usebeamerfont{author} \insertauthor}

            \vspace{1em}
            {\footnotesize
            Orientadores: Prof. Doutor Nuno Miguel da Costa de Sousa Leite \\
            \vspace{-1mm}\hspace{-.5cm}Prof. Doutor Artur Jorge Ferreira}
            
            \vspace{1em}
            {\usebeamerfont{institute} \insertinstitute}
            
            \vspace{2em}
            {\usebeamerfont{date} \insertdate}

            \vspace{-1.5em}\hspace{7.5cm}
            \includegraphics[width=2.5cm]{img/logoisel.png}
        \end{center}
    \end{frame}

    \begin{frame}{Índice}
        \tableofcontents
    \end{frame}

    %------------------------------------------------
    \section{Introdução}
    %------------------------------------------------

    \begin{frame}{Problemas de Agendamento}
        \justifying
        Os problemas de agendamento são problemas de otimização combinatória muito complexos. Estes problemas existem em várias áreas como: Ensino, Medicina, Transporte, Logística, entre outros.

        \vspace{.5em}
        Na área de ensino, os problemas de agendamento podem ser abordados por meio de duas formulações:
        
        \vspace{.5em}
        \begin{itemize}
            \item Agendamento de Disciplinas Baseado no Currículo (CB-CTT)
            \item Agendamento de Disciplinas Após Inscrição (PE-CTT)
        \end{itemize}
    \end{frame}

    \begin{frame}
        Os problemas de agendamento escolar podem ainda ser divididos em três categorias:

        \vspace{.5em}
        \begin{itemize}
            \item Agendamento Escolar (\textit{School Timetabling})
            \item Agendamento de Cursos Universitários (\textit{Course Timetabling})
            \item Agendamento de Exames (\textit{Examination Timetabling})
        \end{itemize}
        \vspace{.5em}

        Este projeto terá um foco na categoria de agendamento de cursos universitários.
    \end{frame}

    %------------------------------------------------
    \section{Estado da Arte}
    %------------------------------------------------

    \subsection{Algoritmos tipicamente utilizados}

    \begin{frame}{Algoritmos tipicamente utilizados}
        \justifying
        Algoritmos tipicamente utilizados na resolução de problemas de agendamento de cursos universitários. 
        \begin{itemize}
            \item Coloração de grafos
            \item Programação Inteira
            \item Procura Tabu
            \item Têmpera Simulada
        \end{itemize}

        Mais recentemente têm-se observado um aumento na utilização de algoritmos híbridos para a resolução deste tipo de problemas. A utilização destes algoritmos tem como objetivo combinar certas propriedades dos algoritmos, o que pode melhorar o seu desempenho.
    \end{frame}

    \subsection{Empresa \textit{Bullet Solutions}}

    \begin{frame}{Empresa \textit{Bullet Solutions}}
        \justifying
        A empresa \textit{Bullet Solutions} é uma empresa tecnológica especializada no desenvolvimento de soluções para otimização de recursos e gestão de horários.

        O projeto será desenvolvido com o objetivo de oferecer uma alternativa competitiva à aplicação disponibilizada pela \textit{Bullet Solutions}, que é atualmente utilizada no ISEL.
    \end{frame}

    \begin{frame}
        \justifying
        Solução disponibilizada pela \textit{Bullet Solutions}:
        \begin{center}
            \includegraphics[width=10cm]{img/exemplo-bullet-solutions-software.png}
        \end{center}
    \end{frame}

    %------------------------------------------------
    \section{Trabalho Realizado}
    %------------------------------------------------

    \subsection{Diagrama de blocos}

    \begin{frame}
        Diagrama de blocos:
        \begin{center}
            \includegraphics[width=.85\linewidth]{img/diagrama-blocos.pdf}
        \end{center}
    \end{frame}

    \subsection{Tarefas realizadas}

    \begin{frame}
        \justifying
        Tarefas realizadas:
        \begin{itemize}
            \item Definição do diagrama de blocos;
            \item Definição dos requisitos funcionais e não funcionais;
            \item Análise detalhada do padrão proposto;
            \item Definição dos casos de utilização;
            \item Definição das restrições dos horários.
        \end{itemize}
    \end{frame}

    %------------------------------------------------
    \section{Trabalho Futuro}
    %------------------------------------------------

    \subsection{Diagrama de Gantt}

    \begin{frame}{Diagrama de Gantt}
        \justifying
        \begin{figure}
            \includegraphics[width=\linewidth]{img/Diagrama-Gantt.png}
            %\caption{Diagrama de Gantt}
        \end{figure}
    \end{frame}

    \subsection{Tarefas Principais}

    \begin{frame}{Tarefas Principais}
        \begin{enumerate}
            \item Implementação de um sistema de importação de dados XML de acordo com o padrão da ITC 2019;
            \item Disponibilizar a importação manual dos dados;
            \item Apresentação de erros ao utilizador caso ocorra alguma falha no funcionamento do sistema;
            \item Apresentação de conflitos no caso de impossibilidade de criação de horários;
            \item Criação de interface gráfica simples e fácil de utilizar;
            \item Permitir ajustes manuais aos horários gerados;
            \item Implementação e otimização dos parâmetros do algoritmo SA;
            \item Implementação de base de dados para a persistência dos dados da aplicação;
            \item Apresentação de \textit{feedback} visual durante o processo de geração de horários.
        \end{enumerate}
    \end{frame}

    \subsection{Tarefas Secundárias}

    \begin{frame}{Tarefas Secundárias}
        \begin{enumerate}
            \item Definição de manchas de disponibilidade através da interface gráfica;
            \item Definição de períodos para aulas partilhadas através da interface gráfica;
            \item Criação de \textit{logs} detalhados do funcionamento do sistema;
            \item Congelamento de salas e/ou professores no momento de geração de horários.
        \end{enumerate}
    \end{frame}

    %----------------------------------------------------------------------------------------

\end{document}